\documentclass[../main.tex]{subfiles}


\begin{document}
\chapter{Introduction}
Wolfgang Pauli initially proposed a weakly interacting, electrically neutral particle to conserve momentum in nuclear decays. The lightest of the fermions, these particles, deemed neutrinos, were often considered to be featureless aspects of the Standard Model, a placeholder that allowed for the conservation for energy, momentum, and eventually lepton flavor.  It also appeared that neutrinos travelled at the speed of light, which requires that they have no mass.  It has been discovered in the last fifty years that a neutrino changes flavor while propagating through space.  The mixing of the flavor and mass neutrino eigenstates creates the mechanism that allows for the neutrino oscillation phenomena.  This requires that the mass eigenstates are differentiable from one another, which implies that at least two of the three neutrino mass eigenstates must have nonzero masses. These neutrino oscillations, whose behavior is not predicted by Standard Model, have elevated neutrino physics as an important avenue in the search for new physics.

Many neutrino experiments in recent decades have explored the parameter space that characterizes the oscillation between flavors.  With each year, experimental results lower the uncertainty on the mixing angles between the mass and flavor eigenstates.  Physicists also continue to examine whether neutrinos gives the leptons a charge parity (CP) violating phase, similar to the CP phase in the quark sector.  These explorations, along with research into the relationship between neutrinos and their antiparticles, can lead to answers of deeper questions like the neutrino’s role in the creation of the matter-antimatter asymmetry of the universe.

Experiments look at the properties of the neutrino by studying their interactions with fundamental particles. An inherent obstacle to neutrino research is the minute strength of the neutrino interaction.  This necessitates the use of large, materially complex detectors for neutrino experiments, as the size of the detector allows for the containment and study of particles generated from the interaction point, while the complexity in terms of the materials presents a greater number of targets due to a larger nucleus.  Designing ever larger detectors with greater complexity increases the difficulty of precisely measuring the neutrino parameters.  Finding the precise cross sections of the neutrino interactions off of various heavy nuclei would help reduce the uncertainties of measurement within these detectors.

The MINERvA (Main INjector ExpeRiment v A) experiment is well positioned to provide these cross sections of neutrino-nucleon interactions in heavy nuclei, as the MINER$\nu$ A detector has a varied number of passive nuclear targets as well as a fine grained active tracker.  The following presents the total and differential cross sections of inclusive pion production from neutrino-nucleon interactions in heavy nuclei in the neutrino energy region of a few GeV. \textbf{Is this still your thesis?}
\section{Neutrino oscillations}
They happen... in space and time

\subsection{PMNS matrix}
Describes the mixing between eigenstates \footnote{A description of
things}

I will include this\cite{lamport94}

%\printbibliography
% This is an example of how you would use tgrind to include an example
% of source code; it is commented out in this template since the code
% example file does not exist.  To use it, you need to remove the '%' on the
% beginning of the line, and insert your own information in the call.
%
%\tagrind[htbp]{code/pmn.s.tex}{Post Multiply Normalization}{opt:pmn}
\end{document}

